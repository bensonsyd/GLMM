
\documentclass{beamer}
\usepackage{graphicx}
\newcommand{\vn}[1]{\mbox{{\it #1}}}
\newcommand{\vb}{\vspace{\baselineskip}}
\newcommand{\vh}{\vspace{.5\baselineskip}}
\newcommand{\vf}{\vspace{\fill}}
\usepackage{tikz}
\usetikzlibrary{arrows}
\newcommand\Fontvi{\fontsize{9}{7.2}\selectfont}


\newcommand{\mypause}{}

\usetheme{Frankfurt}

\title{Generalized Linear Mixed Models and Practical Applications in R}
\author{Sydney Benson, Davita Blyakher, Christina Knudson, Ph.D.}
\institute{University of St. Thomas}
\date{Rose-Hulman Undergraduate Mathematics Conference}

\begin{document}


\frame{\titlepage}

\begin{frame}
\frametitle{Overview}

\begin{itemize}
    \item[] Generalized linear mixed models can overcome limitations of other modeling techniques
    \begin{itemize}
        \item Response distribution
        \item Correlated data
    \end{itemize}
    \item[] Parallel computing can increase the speed of computation
\end{itemize}

\end{frame}

\begin{frame}
\frametitle{Linear Models}

Assumptions:
\begin{itemize}
    \item Independent responses
    \item Normally distributed responses
    \item Responses have equal variances
\end{itemize}

\end{frame}

\begin{frame}
\frametitle{Linear Model Issues}

What if our responses are not normally distributed?
\begin{itemize}
    \item Log odds of your favorite sports team winning a game (binomial)
    \item Log mean number of students per class at your university (Poisson or negative binomial)
\end{itemize}
\vspace{.5cm}
\pause
Use a generalized linear model (GLM)!

\end{frame}

\begin{frame}
\frametitle{Linear Model Issues}

What if our responses are correlated?
\begin{itemize}
    \item Repeated measurements on individuals
    \item Measurements on siblings, parents, relatives
\end{itemize}
\vspace{.5cm}
\pause
Use a linear mixed model (LMM)!
\begin{itemize}
    \item Differences between ``grouped'' data are ``fixed'' effects
    \item Differences within ``grouped'' data are ``random'' effects
\end{itemize}

\end{frame}

\begin{frame}
\frametitle{Linear Mixed Models}

What are "random" effects?
\begin{itemize}
    \item[] Random variables, usually normally distributed with a mean 0
\end{itemize}
\vspace{.5cm}
Random effects are non-observable, but can be estimated by parameters
\begin{itemize}
    \item[] Variance component(s): variance(s) of random effects
\end{itemize}
\vspace{.5cm}
Parameters for linear mixed models:
\begin{itemize}
    \item Fixed effects
    \item Variance components
\end{itemize}

\end{frame}

\begin{frame}
\frametitle{Generalized Linear Mixed Models}

What if responses are not normally distributed \underline{and} correlated? \\
\vspace{.5cm}
Use a generalized linear mixed model (GLMM)!
\begin{itemize}
    \item Combination of GLM and LMM
    \item Include random effects to account for correlation
    \item Model log odds or log mean
\end{itemize}

\end{frame}

\begin{frame}
\frametitle{Salamander Example}

Salamanders: single species from 2 different locations
\begin{itemize}
    \item Rough Butt (R)
    \item White Side (W)
\end{itemize}
\vspace{.25cm}
Do salamanders prefer to mate with others from same location?

\end{frame}

\begin{frame}
\frametitle{Salamander Example}

Correlated: each salamander was mated with multiple others\\
\vspace{.4cm}
Unmeasurable: salamanders have personalized tendencies to mate \\
\vspace{.4cm}
Assumption: each salamander's tendency is independent


\end{frame}

\begin{frame}
\frametitle{Salamander Example}

What affects the probability that a pair of salamanders will mate?
\begin{itemize}
    \item Type of cross (RR, RW, WR, WW)
    \item The female's mating tendency 
    \item The male's mating tendency
\end{itemize}
\pause
\vspace{.5cm}
Fixed effect: type of cross\\
\vspace{.25cm}
Random effect: salamander mating tendencies

\end{frame}

\begin{frame}
\frametitle{Salamander Example} 

Translation to statistical modeling:
\begin{itemize}
\item Response: whether or not the pair mated
\item Fixed effects: $\beta_{RR}, \, \beta_{RW}, \, \beta_{WR}, \, \beta_{WW}$ (log odds of mating)
\item Random effects: each salamander (independent, normal)
\item Variance components: $\sigma_F^2$ and $\sigma_M^2$
\end{itemize}

\end{frame}

\begin{frame}[fragile]
\frametitle{Using \texttt{glmm}}

\begin{verbatim}
    > sal <- glmm(Mate ~ 0 + Cross, 
       random = list( ~ 0 + Female, ~ 0 + Male ), 
       varcomps.names = c( "F" , "M" ), 
       data = salamander,  m = 10^4,
       family.glmm = bernoulli.glmm)
\end{verbatim}

\end{frame}

\begin{frame}[fragile]
\frametitle{Using \texttt{glmm}}

\begin{verbatim}
Fixed Effects:
         Estimate Std. Error z value Pr(>|z|)    
CrossR/R   1.4629     0.2720   5.378 7.53e-08 ***
CrossR/W   0.3781     0.2527   1.496 0.134612    
CrossW/R  -1.7398     0.3157  -5.512 3.55e-08 ***
CrossW/W   1.0345     0.2683   3.857 0.000115 ***
\end{verbatim}

\end{frame}

\begin{frame}[fragile]
\frametitle{Using \texttt{glmm}}

We can translate the log odds back to probabilities:
\begin{align*}
 \text{P(mating)} = \dfrac{\exp \left(\hat{\beta}_{RW}\right)}{1+\exp \left(\hat{\beta}_{RW}\right)}
\end{align*}

\vspace{.1cm}
\begin{tabular}{l|cccc}
Cross & RR & WW& RW & WR  \\ \hline 
Probability of mating &   0.812&0.798& 0.584& 0.149  \\ 
\end{tabular}
\end{frame}

\begin{frame}
\frametitle{Parallel Computing}

\begin{center}
1 person doing 12 calculations \\
\textbf{versus} \\
4 people doing 3 calculations each \\
\end{center}
\vspace{.5cm}
Splitting the computation work of a function amongst the cores available in the computing device \\
\vspace{.25cm}
Most computers have 4 or more cores \\
\vspace{.25cm}
Leads to increased computation speed

\end{frame}

\begin{frame}
\frametitle{Parallel Computing}

How can we do this in R? \\
\vspace{.5cm}
The R package \texttt{parallel}
\begin{itemize}
    \item Detects cores
    \item Divides likelihood approximation calculations amongst cores
    \item Sums results from each core
    \item Returns likelihood approximation
\end{itemize}

\end{frame}

\begin{frame}
\frametitle{Thank you!}
\begin{center}

bens0104@stthomas.edu \\
\url{https://github.com/bensonsyd}\\
\url{https://www.linkedin.com/in/bensonsyd}


\end{center}

\end{frame}

\begin{frame}
\frametitle{References}
\Fontvi
\begin{itemize}
\item[] Knudson C. (2015). \textit{glmm: Generalized Linear Mixed Models via Monte Carlo Likelihood
Approximation.} R package version 1.0.2, URL \url{http://CRAN.R-project.org/package=glmm}.\\ \vspace{.1cm}

\item[] Knudson C. (2016). \textit{Monte Carlo Likelihood Approximation for Generalized Linear Mixed Models.} Ph.D. Thesis, University of Minnesota.\\ \vspace{.1cm}

\item[] Ripley B., Tierney L., Urbanek S. (2017). \textit{Package 'parallel'} R package version 3.3.1, URL \url{http://stat.ethz.ch/R-manual/R-devel/library/parallel/doc/parallel.pdf}. \\ \vspace{.1cm}

\end{itemize}
\end{frame}

\begin{frame}
\frametitle{Using \texttt{glmm}}

Notes:
\begin{itemize}
\item \texttt{0+Cross} produces log odds for each group, without using reference group.
\item \texttt{0+Female} centers random effects for females at 0, likewise for males.
\item Bigger m gives better estimates but takes more time.
\end{itemize}

\end{frame}

\end{document}